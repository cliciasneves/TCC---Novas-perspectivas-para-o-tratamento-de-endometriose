% !TEX TS-program = pdflatex
% !TEX encoding = UTF-8 Unicode

% This is a simple template for a LaTeX document using the "article" class.
% See "book", "report", "letter" for other types of document.

\documentclass[12pt]{article} % use larger type; default would be 10pt

\usepackage[utf8]{inputenc} % set input encoding (not needed with XeLaTeX)

%%% Examples of Article customizations
% These packages are optional, depending whether you want the features they provide.
% See the LaTeX Companion or other references for full information.

%%% PAGE DIMENSIONS
\usepackage[top=3cm,bottom=2cm,left=3cm,right=2cm]{geometry} % to change the page dimensions
\geometry{a4paper} % or letterpaper (US) or a5paper or....
% \geometry{margin=2in} % for example, change the margins to 2 inches all round
% \geometry{landscape} % set up the page for landscape
%   read geometry.pdf for detailed page layout information

\usepackage{graphicx} % support the \includegraphics command and options

\usepackage{titlesec}
\usepackage{lipsum}

\usepackage{indentfirst}

\titleformat{\section}
  {\bfseries}{\thesection.}{1em}{\MakeUppercase}

\titleformat{\subsection}
  {\normalfont}{\thesubsection}{1em}{\MakeUppercase}

\titleformat{\subsubsection}
  {\bfseries}{\thesubsubsection}{1em}{}

\titleformat{\paragraph}[hang]{\normalfont\normalsize\bfseries}{\theparagraph}{1em}{}
\titlespacing*{\paragraph}{0pt}{3.25ex plus 1ex minus .2ex}{0.5em}

\usepackage{tocloft}

\renewcommand{\baselinestretch}{1.50}

%\newlength{\mylen}

%\renewcommand{\cftfigpresnum}{\figurename\enspace}
%\renewcommand{\cftfigaftersnum}{:}
%\settowidth{\mylen}{\cftfigpresnum\cftfigaftersnum}
%\addtolength{\cftfignumwidth}{\mylen}

% \usepackage[parfill]{parskip} % Activate to begin paragraphs with an empty line rather than an indent

%%% PACKAGES
\usepackage[brazilian]{babel}
\usepackage{booktabs} % for much better looking tables
\usepackage{array} % for better arrays (eg matrices) in maths
\usepackage{paralist} % very flexible & customisable lists (eg. enumerate/itemize, etc.)
\usepackage{verbatim} % adds environment for commenting out blocks of text & for better verbatim
\usepackage{subfig} % make it possible to include more than one captioned figure/table in a single float
% These packages are all incorporated in the memoir class to one degree or another...
\usepackage{float}
%%% HEADERS & FOOTERS
\usepackage{fancyhdr} % This should be set AFTER setting up the page geometry
\pagestyle{fancy} % options: empty , plain , fancy
\renewcommand{\headrulewidth}{0pt} % customise the layout...
\lhead{}\chead{}\rhead{}
\lfoot{}\cfoot{\thepage}\rfoot{}

%%% SECTION TITLE APPEARANCE
%\usepackage{sectsty}
%\allsectionsfont{\sffamily\mdseries\upshape} % (See the fntguide.pdf for font help)
% (This matches ConTeXt defaults)

%%% ToC (table of contents) APPEARANCE
%\usepackage[nottoc,notlof,notlot]{tocbibind} % Put the bibliography in the ToC
%\usepackage[titles,subfigure]{tocloft} % Alter the style of the Table of Contents
%\renewcommand{\cftsecfont}{\rmfamiasly\mdseriesakdjdhaskjdasdasdasdasds\upshape}
%\renewcommand{\cftsecpagefont}{\rmfamily\mdseries\upshape} % No bold!

%%% END Article customizations

%%% The "real" document content comes below...

%\title{Novas perspectivas para o diagnóstico clínico da endometriose}
%\author{Clicia Santos Neves}
%\date{} % Activate to display a given date or no date (if empty),
         % otherwise the current date is printed 
\setcounter{secnumdepth}{5}
\setcounter{tocdepth}{5}         
         

\begin{document}


\begin{figure}[h!]
\centering
\includegraphics[width=4.5cm]{ibmr.png}
%\label{Rotu
\end{figure}

\begin{center}
\textbf{CENTRO UNIVERSITÁRIO HERMÍNIO DA SILVEIRA – IBMR \\
LAUREATE INTERNATIONAL UNIVERSITIES \\
CURSO DE BIOMEDICINA}
\end{center}

\vspace{2.5cm}

\begin{center}
\MakeUppercase{\textbf{Clicia Santos Neves}}
\end{center}

\vspace{3.5cm}

\begin{center}
\MakeUppercase{\textbf{Novas perspectivas para o diagnóstico clínico da endometriose}}
\end{center}

\vspace{5.5cm}

\begin{center}
\MakeUppercase{Rio de Janeiro}\\
2015
\end{center}

\newpage


\begin{figure}[h!]
\centering
\includegraphics[width=4.5cm]{ibmr.png}
%\label{Rotu
\end{figure}

\begin{center}
\MakeUppercase{\textbf{Novas perspectivas para o diagnóstico clínico da endometriose}}
\end{center}

\vspace{4.5cm}
Trabalho de conclusão do curso apresentado ao Centro Universitário Hermínio da Silveira IBMR  –Laureate International Universities, como requisito parcial para a obtenção do grau de Bacharel em Biomedicina.

\vspace{2.5cm}

Clicia santos Neves

\vspace{2.5cm}

Orientador: Rômulo Medina Mattos

\vspace{2.0cm}
\begin{center}
\MakeUppercase{\textbf{Rio de Janeiro}}\\
2015
\end{center}

\newpage


\begin{figure}[h!]
\centering
\includegraphics[width=4.5cm]{ibmr.png}
%\label{Rotu
\end{figure}


\begin{center}
\MakeUppercase{\textbf{Clicia Santos Neves}}
\end{center}

\vspace{2.0cm}

\begin{center}
\MakeUppercase{\textbf{Novas perspectivas para o diagnóstico clínico da endometriose}}
\end{center}

\vspace{2.0cm}

Aprovada em julho de 2015 por:

\vspace{9.0cm}

\begin{center}
\MakeUppercase{Rio de Janeiro}\\
2015
\end{center}

\newpage

\begin{center}
\MakeUppercase{\textbf{Agradecimentos}}
\end{center}

\newpage
\begin{center}
\MakeUppercase{\textbf{Resumo}}
\end{center}

\newpage
\begin{center}
\MakeUppercase{\textbf{Abstract}}
\end{center}



\newpage

\listoffigures



\newpage

\tableofcontents










\newpage







%\maketitle

\newpage


\section{Introdução}
A endometriose é uma doença benigna, que acomete mulheres em idade reprodutiva e é causada pela presença do tecido do endométrio fora do revestimento do útero. Ela pode ser encontrada em qualquer local da cavidade abdominal, com maior frequência na pelve, nas estruturas próximas ao útero e raramente acomete outras regiões mais distantes como pulmão,tórax e pericárdio (King, 2007) .

Raramente a endometriose pode apresentar características semelhantes a tumores, como a invasão e a metástase (Robins e Cotran, 2010).

Acredita-se que cerca de 10 a 15\%  das mulheres em idade reprodutiva sejam afetadas por esta doença e que o diagnóstico acontece principalmente em mulheres com dores pélvicas e inférteis (Matta e Muller, 2005).
Segundo Cox e Cols, 2002 a endometriose lidera as causas de infertilidade entre mulheres com idade acima de 25 anos, com isso pode ser possível que aproximadamente 40\% das mulheres inférteis tenham algum grau de endometriose .

Alguns fatores importantes devem ser levados em consideração para a determinação do fenótipo da doença, como a quantidade do fluxo menstrual, fatores genéticos e ambientais (Viganò e Cols, 2012).

Apesar de ser uma das doenças mais investigadas atualmente, a endometriose ainda é de etiopatogenia desconhecida e seu diagnóstico bastante controverso.  Por apresentar sintomas clínicos como dor pélvica crônica, dismenorreia, dispareunia e subfertilidade torna se dificil seu diagnóstico , pois as mulheres com tal condição só procuram ajuda médica quando já se tem a doença estabelecida (Verkauf, 1897 ).Atualmente, não existe um tratamento padrão para endometriose, portanto, o tratamento é individualizado para cada paciente.

O diagnóstico precoce da endometriose é muito importante para garantir uma qualidade de vida melhor para a paciente e evitar o agravamento da doença. Porém uma pesquisa realizada por (Kelechi e Cols, 2009) verificou que há uma demora de cerca de sete  anos entre o aparecimento dos primeiros sintomas até um diagnóstico preciso.

Para se ter um diagnóstico definitivo é preciso uma visualização direta das lesões por laparoscopia ou laparotomia combinado com a histologia de confirmação. Entretanto muitas mulheres deixam de fazer esse exame por considerar uma abordagem invasiva e desnecessária (Kennedy e Cols 2005).
Sendo assim outras técnicas podem são utilizadas para a visualização das lesões como a Utrassom Transvaginal e a Ressonância Magnética .

Atualmente não há consenso médico sobre as causas que levam o tecido endometrial a se desenvolver em outros locais a de sua origem . Entretanto, diversos estudos sobre as características das mulheres que têm a doença ajudam a medicina a se aproximar de maiores respostas permitindo identificar sub-tipos de endometriose. Desta forma, o presente trabalho pretende expor as formas de diagnóstico mais utilizadas atualmete e discutir os recentes diagnósticos para a endometriose.

\section{Objetivo geral}

A endometriose é uma das doenças ginecológicas mais investigadas atualmente, porém ela ainda é uma doença misteriosa, sem cura e pouco conhecida e compreendida pelas mulheres.
Por ter um diagnóstico bastante controverso sua incidência varia muito por isso não há uma forma de prevenção para a doença.

Visando avaliar seu difícil diagnóstico, este estudo tem o objetivo de expor um melhor entendimento desta patologia onde discutiremos as formas de diagnóstico clínico que são utilizadas atualmente e as novas pesquisas para uma visão mais ampla sobre a endometriose na busca por um diagnóstico precoce e minimamente invasivo .

\subsection{Objetivos específicos}

Descrever a fisiopatologia da endometriose.

Demonstrar atuais formas de diagnósticos e tratamento.

Discutir novas perspectivas para um diagnóstico precoce e menos invasivos.
\section{Revisão bibliográfica}

\subsection{Aparelho Reprodutor Feminino}
O aparelho reprodutor feminino é um conjunto de órgãos muito importantes na vida da mulher pois eles são responsáveis pela reprodução, produção dos gametas e dos hormônios e preparação do endométrio para a fecundação e alojamento do embrião caso o óvulo seja fertilizado, caso a fecundação não ocorra o endométrio se descama e ocorre a menstruação. É composto por vários órgãos cujo os mais importantes são: os ovários, a tuba uterina, útero e a vagina (Guyton e Hall.,2006).


\begin{figure}[h!]
%\centering
\includegraphics[width=16cm]{utero.jpg}
\caption[Anatomia do sistema reprodutor feminino]{Anatomia do sistema reprodutor feminino (Disponível em Guyton e Hall,2006 pag. 1012)}
%\label{Rotulo}
\end{figure}

\subsection{útero}



Há quem acredite que o útero tem como função somente abrigar o feto no período gestacional. Porém já se sabe que esse órgão tem papel importante no ciclo ovulatório e é fundamental no equilíbrio da vida da mulher.

O útero é um órgão muscular e glandular, possue uma cavidade interna e mede cerca de 8 cm de comprimento, 5 cm de largura e 3 cm de espessura . Seu formato é semelhante a uma pêra invertida e está subdividido em três partes: fundo (região bem acima), corpo e colo ou cérvix.  

A parede uterina é formada por três camadas: a mais externa constituída por mesotélio e tecido conjuntivo, o miométrio que é constituído de músculo liso e o endométrio ou mucosa uterina (Junqueira e Carneiro, 2008). Endométrio é um tecido vascularizado que reveste a parede interna do útero e tem como função acolher e nutrir o embrião na gravidez, sendo assim durante o período pré -puberal o tecido endometrial permanece inativo. Ele é composto por glândulas e estroma e sofre alterações fisiológicas e morfológicas em resposta aos hormônios ovarianos estrogênio e progesterona durante o ciclo menstrual (Filho, 2011).
Desta forma, o útero passa por esse processo cíclico normalmente a cada trinta dias em mulheres com idade reprodutiva.

Há diversas teorias que tentam explicar como o endométrio vai parar fora do útero. Uma delas é que durante a menstruação células do endométrio invadem as trompas e vão parar em vários lugares do abdomem ( Robbins e Cotran, 2010 ).


\subsection{Ciclo menstrual}

\begin{figure}[h!]
%\centering
\includegraphics[width=16cm]{ciclo.jpg}
\caption[Alterações morfológicas e hormonais durante o ciclo menstrual]{ Alterações morfológicas e hormonais durante o ciclo menstrual ( disponível em ) } 
%\label{Rotulo}
\end{figure} 


Toda mulher em seu estado fisiológico normal em idade reprodutiva entre a menarca e a menopausa , sofre alterações cíclicas em seu sistema reprodutivo conhecido como ciclo menstrual. O ciclo menstrual causa alterações em diversas partes do corpo feminino incluindo mamas, útero, pele e ovários. Ele envolve um conjunto de reações quimicas endócrinas que afeta aspectos biológicos e sociológicos como a reprodução, sexualidade e fertilidade da mulher. Todo o ciclo dura em média 28 dias e se caracteriza pela secreção dos hormônios femininos que vão agir para a liberação do óvulo e preparação do endométrio uterino para a implantação do óvulo caso seja fertilizado.

O sistema hormonal feminino (Figura 2) é seguido por três hierarquias de hormônios a serem liberados: O hormônio liberador de gonadotrofinas (GnRH) secretado pelo hipotálamo, o hormônio folículo estimulante (FSH) e o  hormônio luteinizante (LH) que serão liberados pela hipófise anterior em resposta à liberação do GnRH e por fim os hormônios ovarianos estrogênio e progesterona, ambos secretados pelos ovários em resposta à liberação do FSH e do LH (Guyton e Hall, 2006 ).

O endométrio sofre modificações tanto histológicas quanto morfologicamente em resposta à liberação do estrogênio e da progesterona liberados pelos ovários durante o ciclo menstrual (Geraldo Brasileiro Filho , 2011). Essas mudanças que tem início a cada ciclo, tem a finalidade de preparar a parede uterina para a nidação do blastocisto caso haja a fecundação.

A primeira alteração conhecida como fase proliferativa, ocorre simultâneamente a fase folicular do cilo ovariano (Mattson e Matfin, 2010).

No início de cada ciclo o endométrio  se descama preservando as células de sua camada mais profunda. Sob a influência do estrogênio liberado pelos ovários, estas células endometriais da camada mais basal se proliferam rapidamente tanto as glândulas quanto o estroma e em poucos dias a superfície do endométrio é reepitelizada cessando a descamação e o sangramento (Curi e Procopio, 2009).


A fase secretora acontece após a ovulação. Nos ovários o corpo lúteo passa a produzir estrogênio e principalmente progesterona. A progesterona vai atuar no endométrio estimulando a secreção das glândulas (Mattson e Matfin, 2010) e fazendo com que o endométrio fique ainda mais vascularizado . A medida em que a progesterona vai predominando em relação ao estrogênio, a capacidade proliferativa do endométrio diminui sua velocidade e a atividade mitótica é interrompida (Karp e Jason, 2015).


Aproximadamente uma semana após a ovulação o endométrio encontra se bastante vascularizado repleto de glicogênio, as glândulas endometriais com sua atividade secretora máxima pronto para receber o óvulo fecundado. Com a não ocorrência da fecundação, o corpo lúteo regride e o endométrio sofre a descamação dando início a um novo ciclo (Curi e Procopio, 2009).



\section{Endometriose}
A endometriose é uma doença ginecológica comum, de caráter benigno que se caracteriza pela presença de tecido endometrial funcional com actividade celular evidente em lesões, nódulos, quistos ou endometrioma (Audebert et al., 1992) fora da cavidade uterina ou seja: ovários, ligamentos uterinos, septo vaginal, fundo de saco, peritônio pélvico, intestino grosso e delgado (Robins e Cotran, 2010). Embora esses sejam os locais mais comuns, a endometriose pode raramente acometer órgãos e tecidos  fora da pélvis como bexiga , trato urinário e outros locais como pulmão, mamas e ossos (Filho , 2012) .
( colocar uma imagem de uma lesao endometriótica) 


A endometriose é uma condição crônica estrogênio dependente que geralmente acomete mulheres em idade reprodutiva (Kumar e Cols , 2012) . Apesar de ser uma doença comum e frequentemente estudada, sua causa ainda permanece desconhecida, havendo um numero cada vez maior de casos diagnosticados. Estima-se que a endometriose afeta de 10 a 15 \% das mulheres na pré menopausa e de 15 a 70\% em mulheres com infertilidade pre diagnosticada. Algumas condições clinicas, genéticas e imunes podem ser ditas como fatores de risco para o aparecimento  das lesões endometriais como: o aumento da dor e do fluxo menstrual, menarca precoce e até mesmo algum parente de primeiro grau com a mesma condição patológica (Porth e Matfin, 2010).
As lesões endometriais ectópicas possui tecido ativo que responde de forma semelhante ao endométrio eutópico durante o ciclo menstrual por responder aos estímulos dos hormônios ovarianos estrogênio e progesterona causando hemorragias, inflamação e cicatrizes (Dongxu e Cols, 2014).

Embora as lesões endometrióticas  normalmente apareça com aspectos benígnos na histopatologia, de acordo com alguns autores, a endometriose tem sido comparado a um tumor maligno já que as lesões tem a capacidade de crescer, se infiltrar e aderir aos tecidos adjacentes (Kitawaki e Cols, 2002;. Noble e Cols, 1996 ). Porém ela raramente apresenta essas características de metástase e invasão (Robins e Cotran 2010).

Entretanto, a endometriose sendo uma doença bastante complexa, provavelmente alguns grupos de fatores distintos quando combinados podem ter um papel importante na fisiopatologia da doença tais como polimorfismo genético, interferência hormonal através de receptores de estrogênios e progesterona , fatores do sistema imunológicos e endócrinos (troncon e Cols , 2014).

\subsection{classificação das lesões endometrióticas}


\subsubsection{Quanto a coloração}

Cada lesão endometrial pode caracterizar-se macroscopicamente de forma diferente variando na coloração, na sua forma e na localização, podendo ser típica, quando apresenta cor escura ou atípica caracterizada pela cor amarelada ou avermelhada indicando maior atividade da doença.

Os aspectos das lesões sofrem uma variação de cor que vai da coloração inicial amarelada para a  "café com leite" até alcançar uma coloração escura. Essa variação ocorre devido ao acúmulo de hemossiderina no foco da lesão. Redwine, 1987 , relatou que essas alterações entre a mudança de claro até chegar ao escuro ou cicatricial, levam em média de 7 a 10 anos.

No ovário o tecido endometrial pode formar cistos , conhecidos como cistos de chocolate pois eles estão repletos de sangue velho que se assemelha a calda de chocolate (Figura 4). Em outras partes da pelve o tecido pode assumir a forma de pequenas lesões hemorrágicas que podem ser negras , azuladas, vermelhas, claras ou opacas (Porth e Matfin ,2010).

As lesões vermelhas são altamente vascularizadas , causando sangramento para dentro da cavidade peritoneal, aderências e inflamação. Já as lesões mais escuras, ou as brancas, representam uma lesão cicatricial da endometriose.Do ponto de vista histológico, essas lesões são pouco vascularizadas e estão correlacionadas com uma maior quantidade de fibrose (Brosens , 1997).

\begin{figure}[h!] 
%\centering
\includegraphics[width=16cm]{endometriomaroto.jpg}
\caption[Endometriose: implantes do endométrio no ovário] {implantes do endométrio no ovário ( disponível em )}
%\label{Rotulo}
\end{figure} 


\subsubsection{Quanto a diferenciação histológica}

Porto e Cols 2015, caracterizaram o tecido endometrial da seguinte forma (Figura 4): Estromal (tecido com presença de estroma com aspectos morfológicos similares ao endométrio tópico) ; Glandular bem diferenciada (caracterizada por conter células epiteliais com morfologia semelhante ao endométrio tópico); Glandular indiferenciada (Tecido com epitélio glandular plano ou cubos baixos sem correspondência com o endométrio ectópico) ; Mista (Tecido no qual possui a presença de glândulas de padrão diferenciado e indiferenciado). A partir desse estudo constataram que o tipo histológico glandular bem diferenciado  predominam nas lesões superficiais , já os tipos glandulares indiferenciados e mistos tem uma maior prevalência nas lesões endometrióticas profundas que acomete o peritônio e o intestino.


\begin{figure}[h!]
%\centering
\includegraphics[width=16cm]{citoendometriose16:36.jpg}
\caption[Amostras histológicas de diversos tipos de endometriose]{Lâminas mostrando padrão histológico dos diversos tipos de endometriose. (A) endometriose estromal ; (B) endometriose glandular bem diferenciada ; (C) endometriose glandular indiferenciada; (D) endometriose glandular mista.( Disponível em Porto e Cols , 2015 )}
%\label{Rotulo}
\end{figure} 


\subsubsection{Quanto a infiltração}



Os implantes endometriais de acordo com seu crescimento e evolução , podem apresentar se de formas distintas podendo ser classificadas como superficiais ou peritoneais, quando as lesões são muito pequenas com o tamanho de até 2mm e geralmente não aparcem em exames de imagem ; intermediária as lesões que atingem de 2 a 4 mm ; profundas os implantes endometriais que atingem uma área igual ou superior a 5mm, essas lesões geralmente são compostas por células musculares, epitélio glandular ativo, estroma escasso, com reação inflamatória e pode causar fibrose nos tecidos adjacentes (American Society for Reproductive Medicine, 1997).

Koninckx e Martin, 1992 , descreveu a endometriose da seguinte forma:

$\bullet$ Tipo I - É caracterizada por uma lesão com maior diâmetro na superfície peritoneal, que habitualmente se apresenta com aspectos esbranquiçados.

$\bullet$ Tipo II - Nas lesões peritoneais , o reto encontra-se aderido ao fundo de saco de Douglas e ligamentos uterossacros.

$\bullet$ Tipo III - o implante endometrial caracteriza-se como tendo seu maior diâmetro abaixo da superfície do peritôneo , de forma esférica e sua localização é no septo reto-vaginal que aparecem como nódulos palpáveis. 

\subsubsection{Quanto a severidade e o comprometimento dos órgãos}

De acordo com Acosta e Cols, 1973, a endometriose pode ser:


$\bullet$ Endometriose leve: lesões não associados com cicatrizes e retração do peritôneo no fundo de saco anterior ou posterior ou no peritôneo pélvico.


$\bullet$ Endometriose moderada: Afeta um ou ambos os ovários com diversas lesões superficiais, com formação de cicatrizes ou pequenos implantes endometriais. Aderências periovarianas mínimas associadas a lesões do ovário.


$\bullet$ Endometriose acentuada: lesões afetando um ou os dois ovários com endometriomas superior a 2 cm. Lesões afetando uma ou ambas as trompas, obstruídas por implantes endometriais com aderência ou lesão associada a endometriose.



\subsection{Teorias}


A fisiopatologia da endometriose permanece desconhecida, embora muitas teorias foram desenvolvidas para tentar justificar a presença do tecido endometrial fora do seu local de origem.

\subsubsection{Teoria da menstruação retrógrada}


Até hoje a teoria da menstruação retrógrada é a teoria mais aceita e foi descrita por Sampson em 1920 , onde ele afirmou que a implantação do tecido do endométrio fora da cavidade uterina se dá através da regurgitação do sangue menstrual pelas trompas de falópio e indo parar no peritônio. Estas células endometriais provocam reação inflamatória e dor nos períodos menstruais. Observou-se que a incidência de endometriose é maior em mulheres que tem refluxo tubário , porém nem todas as mulheres que possuem menstruação retrógrada sofrem de endometriose , levando a crer que a patogenicidade da doença pode ser um mecanismo multifatorial (Macer e Taylor, 2013).

\subsubsection{Teoria da metaplasia celômica}

A teoria da metaplasia celômica foi proposta por Mayer em 1919 e retificada por Nisolle e Cols, 1997. Mayer propôs essa teoria com base no fato de que o peritôneo e endométrio eutópico tem a mesma origem embrionária. Deste modo, estes tecidos mediante a estímulos hormonais teriam a capacidade de se diferenciar em células estromais e glandulares características do epitélio endometrial eutópico, sugerindo assim a origem dos focos de endometriose.
.

\subsubsection{Teoria da disfunção imunológica}

Esta teoria ajuda explicar o por que mulheres com menstruação retrógrada desenvolvem endometriose e outras não. Pois pesquisas realizadas verificaram que mulheres com endometriose tem imunidade alterada suportando a possibilidade de que a patogênese da endometriose pode estar envolvida com uma resposta imune deficiente (Sinaii e Cols, 2002).

Pacientes com endometriose tem uma concentração mais elevada de macrófagos ativados,diminuição da imunidade celular e repressão das células NK (Sikora e cols, 2011 ; Osuga e Cols, 2011).      

As lesões endometrióticas desencadeia uma resposta inflamatória recrutando macrófagos e leucócitos ativados no local(Kyama e Cols, 2009). Esta resposta inflamatória pode impedir a eliminação dos resquícios menstrual provenientes da menstruação retrógrada e consequentemente promover a implantação e o crescimento das células do endométrio nos locais ectópicos(Christodoulakos e Cols, 2007). 

De acordo com (Ulukus e Arici, 2005) tanto as células do sistema imune  quanto as células endometriais secretam citocinas e fatores de crescimento os quais induzem a proliferação celular e a angiogênese promovendo assim a implantação e o crescimento das lesões endometriais.




\subsection{Sintomas} 


Os aspectos clínicos da doença consistem em dismenorreia grave, dispareunia e dor pélvica em decorrência de sangramentos e aderências intrapélvicas.Dores durante a defecação indica envolvimento da parede retal e a disúria no envolvimento da bexiga. Quando a doença afeta o intestino delgado podem ocasionar em perturbações intestinais ( Robins e Cotran , 2010).

A sintomatologia da endometriose  é bastante variada e autores afirmam que não há relação entre os aspectos das lesões e os sintomas da doença ( martin e cols , 1989 ).
De acordo com um estudo realizado por Hedyeh Riazi e cols,2015, os sintomas associados a esta patologia são diversos e varia de acordo com o diagnóstico clinico de cada paciente .Os sintomas mais comuns relatados da endometriose são: Dores que se agravaram ao longo do tempo descritas como dores pélvicas durante o ciclo menstrual, dores ao ter relações sexuais, dor lombar, dor retal ao evacuar no período menstrual,infecções no trato urinário,inchaço , enxaqueca,cansaço , febre, sangramentos e infertilidade.Distúrbios digestivos, náuzeas , dores no estômago, candidíase, doença inflamatória pélvica,cistos ovarianos,  tonturas, fadiga crônica, também são sintomas que foram foram atribuídas a endometriose.

Quando os implantes endometriais estão presentes na bexiga, os sintomas característicos são hematúria e disúria. A hematúria cíclica é uma característica importante da endometriose na bexiga porém só está presente em 20\% das pacientes (Akpınar e Cols, 2015).

Segundo Gao X e cols,2006  pacientes com endometriose podem desenvolver transtornos psicológicos como ansiedade e depressão.


Deste modo, estes sintomas associados geram impacto negativo no bem estar físico, mental e social da mulher impedindo-a de realizar atividades de sua vida cotidiana, diminuindo sua qualidade de vida e desempenho sexual ( Fourquet e Cols, 2011).


A endometriose também pode ser assintomática, como a endometriose intestinal que aparece como implantes serosos e ocorre em 3\% a 37\% dos pacientes com endometriose (Pisanu e Cols, 2010).


Os pacientes com endometriose tendem a desenvolver sintomas adicionais, tais como alergias, fibromialgia, asma, eczema, doença inflamatória autoimune, síndrome da fadiga crônica e hipotireoidismo (Sinaii e Cols, 2002).

Há também um maior risco de câncer de mama em mulheres diagnosticadas com endometriose após os 40 anos, devido à sua maior exposição ao estrogênio endógeno elevado (Bertelsen e Cols, 2007). 


\subsection{Diagnóstico}

A endometriose é uma doença debilitante que pode causar vários problemas na vida da mulher, entretanto quando diagnosticada precocemente ela pode ser tratada e seus sintomas diminuídos.
Embora o diagnóstico definitivo da endometriose necessite de uma intervenção cirúrgica, preferencialmente por videolaparoscopia, diversos achados nos exames físico, de imagem e laboratoriais já podem predizer, com alto grau de confiabilidade, que a paciente apresenta endometriose.



\subsubsection{Diagnóstico Clínico}

A investigação inicial da endometriose é feita por meio de uma anamnese para avaliação dos sintomas clínicos. Vale lembrar que o grau de severidade da doença não está relacionado com os sintomas . Entretanto, o diagnóstico da endometriose se torna difícil de se confirmar devido a variabilidade de sintomas , bem como a correlação não confiável entre a apresentação clínica e os achados cirúrgicos (Wellbery, 1999) pois esses sintomas podem estar relacionados com outras condições patológicas como a síndrome do intestino irritável e doença inflamatória pélvica. Por isso muitas vezes há um atraso de vários anos entre o início dos sintomas e o diagnóstico definitivo (Hadfield e Cols, 1995 ; Arruda e Cols 2003). Segundo Kundu e Cols, 2015 esse atraso no diagnóstico se dá também pela má relação entre médico e paciente, pois de acordo com relatos dos paciente portadoras da doença, muitos médicos não levam a sério as condições clínicas do paciente dificultando seu diagnóstico.


Há uma grande variedade de sintomas dolorosos que podem estar relacionados à aparência, ao grau de invasão, à localização e à profundidade de acometimento das lesões. São seis os sintomas que devem ser investigados: dismenorréia, dispareunia, dor pélvica acíclica, infertilidade e alterações urinárias e intestinais cíclicas.

Os sinais e sintomas associados ao exame clínico ginecológico já podem dar uma ideia da presença da doença e do comprometimento dos órgãos pélvicos afetados, como é o caso da endometriose profunda infiltrativa, cujos sinais sugestivos são nodulações palpáveis no fórnice vaginal posterior ou septo retrovaginal, espessamento dos ligamentos uterossacros ou lesões na vagina (Kennedy e Cols, 2005).  

Esterilidade, irregularidade menstrual e dispaurenia (profunda) são outras queixas. Em princípio, qualquer sinal ou sintoma que surja ou se agrave no período menstrual tais como hematúria (menúria), disúria, urgência miccional, polaciúria, dor ou sangramento à evacuação, diarréia, deve nos sugerir diagnóstico de endometriose. O exame físico pode ser aparentemente normal ou revelar nódulos (vagina, colo, septo reto-vaginal...), espessamento (paramétrio), útero com mobilidade reduzida, e doloroso à mobilização; massas anexiais.

Se o paciente estiver com sintomas de dor sugestivo de endometriose deverá ser tratado sem um diagnóstico definitivo, e exames adicionais deverá ser solicitado.

 
\subsubsection{Diagnóstico laboratorial}

Podemos utilizar também exame de sangue para pesquisa da presença do CA 125. O CA 125 é um marcador da endometriose, ou seja, ele esta presente na corrente sanguínea em níveis superiores aos normais, quando a paciente está com doença em atividade, e se ele estiver com nível superior a 100 U/ml, é indicativo de endometriose avançada muitas vezes já com comprometimento intestinal. 

O CA-125, anticorpo monoclonal contra antígenos do epitélio ovariano. Não é específico podendo surgir também em outros tumores de linhagem epitelial em enfermidades hepáticas, etc. Na endometriose avançada pode se elevar a níveis superiores a 100U/ml. Deve ser dosado no soro no período menstrual.

\subsubsection{Laparoscipia}


Considerado padrão ouro, a laparoscopia é uma técnica cirúrgica eficaz com o objetivo de excisão endometriose visível e confirmação através de exame histológico. Embora a comparação citológica seja padrão para o diagnóstico da endometriose, algumas lesões visualizadas durante a cirurgia não são examinadas em biópsia por que a visualização direta é suficientemente característica (Adamson e Nelson, 1997).Como a endometriose é uma condição crônica, é comum haver reincidência.



A laparoscopia é um procedimento cirúrgico onde o paciente é anestesiado com uma anestesia geral, uma pequena incisão é feita perto do umbigo, e um laparoscópio (tubo iluminado flexível) é inserido nessa incisão. Com o laparoscópio é possível examinar e procurar possíveis implantes endometriais no abdome órgãos abdominais do paciente com possíveis implantes endometriais no abdome  do paciente.  


É importante notar que nem todas as lesões de endometriose são visualizados no momento da cirurgia; lesões podem estar escondidos por órgãos ou aderências pélvicas, microscópico ou semelhantes na aparência com outras lesões malignas ou benignas, como o câncer de ovário, hemangioma ou gravidez ectópica (Brosens e Cols, 2004). 


\subsubsection{Diagnóstico por métodos de imagem}

\paragraph{Ultrassonografia } 

Por ser de custo acessível, ausência de radiação ionizante e fácil acesso, a ultrassonografia transvaginal é o primeiro exame de imagem a ser solicitado para a paciente com história e exame físico sugestivo de endometriose (Fleischer e cols, 1996), preferencialmente com preparo intestinal, que tem por objetivo eliminar os resíduos fecais e gases eventualmente presentes no intestino, atrapalhando a identificação das lesões de endometriose encontradas no reto, sigmoide regiões retrocervical e septo retovaginal. Um estudo de Abrão e Cols,2007 , avaliando o exame, demonstrou uma sensibilidade de 94\% e uma especificidade de 98\% na identificação de focos de endometriose profunda.

Bazot e Cols, 2003, demonstrou que a ultrassonografia transvaginal foi tão eficiente quanto a transretal endoscópica, tanto para o diagnóstico quanto para a avaliação da extensão da endometriose retal.

Para avaliação de endometriose intermediária ou profunda, a ultrassonografia transvaginal é um método eficiente (Moore e Cols, 2012). Como relatado por Fedele e Cols, 1997, essa técnica é recomendada para o diagnóstico de endometriomas e endometriose da bexiga, porém quanto a avaliação de lesões superficiais peritoneais ou focos de endometriose no ovário, o uso da ultrassonografia como diagnóstico ainda é inconclusivo.

Segundo Bazot e Cols, 2003, a principal limitação das técnicas de ultrassonografia é pela falta de amplitude da área estudada, pois elas se concentram em uma área anatômica limitada da cavidade pélvica.



\paragraph{Ressonância Nuclear Magnética (RNM)}

A ressonância nuclear magnética (RNM) é recomendada para complementar o diagnóstico da endometriose pélvica. As principais indicações estão relacionadas ao estadiamento da doença, ou seja, na maneira de avaliar a extensão da endometriose em relação ao órgão no qual se originou, e na análise de imagens duvidosas no estudo com o ultra-som.

A RM recentemente foi usada para o diagnóstico em que os ligamentos sacro uterinos revelou falta de sensibilidade para o diagnóstico da doença com envolvimento retal (Kinkel e cols, 1999).

No estudo realizado por Bazot e Cols, 2003, com pacientes que foram submetidos a RNM, eles obtiveram uma sensibilidade de 90,3\% e especificidade de 91\% . Também fizeram uma comparação  de RNM e achados cirúrgicos patológicos de pacientes com endometriose pélvica comprovada onde foram observadas áreas de tecidos correspondente a fibrose, característico de endometriose. Quando a RNM foi comparada com a ultrassonografia transvaginal, a RNM mostrou se mais específica pois era capáz de fazer uma avaliação mais ampla da pélvis. Porém quando a endomeriose pélvica profunda está localizada no cólon sigmóide, pode ser confundido com material fecal, sendo assim, o uso de enema antes da RNM é útil para o diagnóstico neste local (Kinkel e cols, 1999).





\paragraph{Eco-colonoscopia}

Indicado especialmente em casos de suspeita de endometriose envolvendo o intestino. Baseia-se na realização de um ultra-som por via transretal, associado a um exame de colonoscopia. Desta forma, tem-se uma imagem que pode indicar a presença da endometriose e permite, ainda, avaliar as camadas de intestino acometidas pela doença. No entanto é um exame mais invasivo quando comparado a outros exames como o ultra-som com preparo intestinal e a ressonância, além de estar limitado à avaliação apenas do intestino


\subsection{Perspectivas para um novo diagnóstico da endometriose}














\subsection{Tratamento} 
Tratamentos para a endometriose é individualizado e vai depender da intensidade dos sintomas e da evolução das lesões. sendo assim o tratamento pode envolver tanto a intervenção farmacológica ou cirúrgica , lembrando que dependendo do paciente e do estágio das lesões esses procedimentos são feitos de forma integrada(Hernando e Cols,2015). Os tratamentos médicos farmacológicos disponíveis para a endometriose incluem contraceptivos orais, análogos de GnRH, análogos de progesterona e inibidores da aromatase, cada um desses medicamentos tem suas especificidades e cabe ao médico avaliar a gravidade da doença para prescrever a melhor forma de tratamento para cada caso individualizado (Braundmeier e Fazleaba, 2009).

A forma de tratamento para a endometriose pode ser dividida em três categorias: alívio da dor, supressão endometrial e cirurgia( Mattson e Matfin ,2010 ). A escolha do tratamento dependerá da gravidade dos sintomas, da extensão e localização da doença, do desejo de engravidar e da idade da paciente.

Ainda há uma necessidade de novos medicamentos para tratamento da endometriose que reduza a dor pélvica , minimize a intervenção cirúrgica e preserve a fertilidade (Hernando e Cols,2015).


A endometriose pode retornar independente do tratamento( exceto a cirurgia radical) e esse retorno pode estar associada com a gravidade da doença. Com o tratamento clínico as taxas de recidivas da doença depois de 7 anos variam de 34 \% em mulheres com doença leve até 74\% em mulheres com a doença grave. Foram relatadas taxa de recidivas de 20 a 40\% dentro de 5 anos após cirurgia (Mattson e Matfin ,2010) .


\subsubsection{Análogos de GnRH}

Sabe-se que o GnRH (hormônio liberador de gonadotrofinas) é um hormônio liberados de forma pulsátil em média de 5 a 25 minutos de duração secretado pelo hipotálamo que age sobre a hipófise levando como resposta a liberação de LH e FSH (hormônio luteinizante e hormônio folículo estimulante respectivamente). Ele é responsável por regular indiretamente a atividade gonadal por meio da liberação do LH e do FSH ( Guyton e Hall,2006 ). Deste modo, os antagonistas de GnRH tem como principal mecanismo de ação inibir os receptores competitivos de GnRH , já os agonistas de GnRH provocam uma liberação inicial de gonadotrofinas seguida de uma dessensibilização gonadotrópica e desregulação da sinalização intracelular. sendo assim, Huirne  e Lambalk , 2001, mostrou que a utilização de antagonistas de GnRH deve possivelmente ser tão eficaz quanto a utilização de um agonista de GnRH no tratamento da endometriose.
Alguns fármacos mais utilizados como análogos de GnRH no tratamento da endometriose são a Leuprolida e a Goserelina ( Roladex ) eles são eficazes no melhoramento da dor das pacientes com endometriose , no entanto, essas medicações tem provocado efeitos adversos , como suores noturnos e a diminuição significativa de massa óssea (Schrager e Cols, 2013). 

\subsubsection{Danazol}

É um fármaco utilizado no tratamento da endometriose pois se mostrou eficaz no combate a dores pélvicas associadas endometriose. Ele pode ser administrado tanto por via oral ou por via vaginal através de anéis vaginais, cremes ou dispositivos intrauterinos, porém sua utilização ainda é limitado por causar efeitos colaterais androgenicos significativos como ganho de peso, acne, seborreia e pelos excessivos pelo corpo(hirutismo) .

Estudos realizados por Cottreau e cols 2003, mostrou uma associação entre o uso do danazol e o aparecimento de câncer no ovário. Os autores sugerem que os andrógenos circulantes com o uso do danazol podem estar envolvidos no desenvolvimento do câncer ovariano pois foi observado uma grande expressão de receptores androgênicos na maioria dos tumores no ovários.

\subsubsection{Inibidores da aromatase ( IAs )}

Como a endometriose é considerada uma doença crônica dependente de estrogênio, inibidores da aromatase são utilizados no tratamento da doença como o Anastrozol e o Letrozol. Aromatazes são enzimas que converte andrógenos adrenais em estrogênio( Almassinokiani e cols, 2014 ), e esses inibidores utilizados tem como mecanismo de ação inibir esta enzima bloqueando a conversão e suprimir os níveis de estrogênio em tecidos de endometriose. Os IAs são utilizados para tratar as dores pélvicas provenientes da endometriose reduzindo os tamanhos das lesões . Ele também mostrou se eficaz quando administrado em conjunto com os análogos de GnRH em mulheres que mostraram resistência aos tratamentos convencionais (Usluogullari e cols ,2015).




\subsubsection{Tratamento cirúrgico}


Quando o tratamento medicamentoso e hormonal não surtem os efeitos desejados, a cirurgia pode ser necessária para aliviar a dor e aumentar possibilidade de gestação. 
O tratamento cirúrgico da endometriose compreende desde procedimentos de baixa complexidade, como cauterização de focos superficiais e liberação de aderências, até intervenções complexas nos ovários, fundo de saco de Douglas, intestino, bexiga e ureteres (Guo, 2009). Além disso, muitas pacientes apresentam infertilidade associada à dor, exigindo que o procedimento cirúrgico seja conservador. Entretanto, é importante que tanto o médico quanto a paciente estejam cientes de que a cirurgia conservadora implica índices elevados de recorrência da dor.

Para a paciente com infertilidade associada a endometriose de grau mínimo e leve, a ablação dos focos e a adesiólise mostrou se eficaz melhorando a fertilidade (Jacobson e Cols, 2002).

Para os pacientes com endometriomas ovarianos e que não respondem adequadamente ao tratamento medicamentoso, a cirurgia indicada nos casos de endometriomas sintomáticos ou
grandes(Chapron e Cols, 2002).


\subsection{reincidência}

Morgante e Cols, 1999, relataram que os pacientes tratados com baixa dose de danazol (100 mg / dia durante 6 meses) após a cirurgia laparoscópica e seis meses de terapia com GnRH agonistas terapia reduziu a incidência de dor pélvica quando comparados com aqueles que não receberam a terapia . Especificamente, 24 meses após a cirurgia, 14 pacientes randomizados para receber tratamento com danazol, a  dor reincindio porém foi significativamente menor do que os outros 14 pacientes que não receberam qualquer tratamento.

\section{Referências} 

COX, H, HENDERSON, L., ANDERSEN,N., CAGLIARINI , G, e SKI.C(2003). Fucus group of endometriosis: Struggle, loss and the medical merry-go-round. Internetional journal of nurseng practice , 9,2-9

GUYTON; Fisiologia Humana; Sexta edição; Rio de Janeiro; Editora Guanabara
Koogan, 2011; 545 p.

GUYTON e HALL; Tratado de Fisiologia Médica; Décima primeira edição; Rio de
Janeiro; Editora Elsevier, 2006; 1011 p.

SILVERTHORN; Fisiologia humana ; Quinta edição; São Paulo; Editora Artmed, 2010; 848 p.

Viganò P, Somigliana E, Panina P, Rabellotti E, Vercellini P, Candiani M.Principles of phenomics in endometriosis.Hum Reprod Update. 2012 May-Jun;18(3):248-59

Verkauf BS. Incidence, symptoms, and signs of endometriosis in fertile and infertile women. J Fla Med Assoc. 1987

Kelechi E. Nnoaham, Sivahami Sivananthan, Lone Hummelshoj, Crispin Jenkinson, Premila Webster, Stephen H. Kennedy, and Krina T. Zondervan. MULTI-CENTRE STUDIES OF THE GLOBAL IMPACT OF ENDOMETRIOSIS AND THE PREDICTIVE VALUE OF ASSOCIATED SYMPTOMS.J Endometr. 2009; 1(1): 36–45. 

Kennedy S1, Bergqvist A, Chapron C, D'Hooghe T, Dunselman G, Greb R, Hummelshoj L, Prentice A, Saridogan E; ESHRE guideline for the diagnosis and treatment of endometriosis.Hum Reprod. 2005 Oct;20(10):2698-704. 


PORTH, M. C. e MATFIN, G.; Fisiopatologia; Oitava edição, volume 2; Rio de
Janeiro; Editora Guanabara Koogan, 2010; 1146 p.

GERALDO BRASILEIRO FILHO; Bogliolo Patologia; oitava edição; Rio de Janeiro; Editora Guanabara Koogan, 2011; 604 P.

THOMAS C. KING; Patologia; Primeira edição ; Rio de janeiro; Editora Elsevier, 2007; 347 P

KARP, JASON R., Marathon e Beyond , 2015, Vol. 19 Issue 2, p112 e 113.

ADRIANA ZANONA DA MATA E MARISA CAMPIO MULLER; Uma análise qualitativa da convivência da mulher com sua endometriose. Psicologia, saúde e doenças. 2006, 7 (1), 57-72

LAURA KNABBEN,SARA IMBODEN, BERNHARD FELLMANN,KONSTANINOS NIRGIANAKIS,ANNETTE KUHN, MICHAEL D. MUELLER. Urinary tract endometriosis in patients with deep infiltrating endometriosis: prevalence, symptoms, management, and proposal for a new clinical classification, 2014.

LENE N. HEIDEMANN , DORTHE HARTWELL , CHRISTIAN H. HEIDEMANN e KIRSTEN M.
JOCHUMSEN ; The relation between endometriosis and ovarian cancer – a
review;Acta Obstetricia Et Gynecologica Scandinavica ;2013

Sun Mi Hwang, Chung Won Lee, Byung Seok Lee, and Joo Hyun Park ; Clinical features of thoracic endometriosis: A single center analysis, 2015.

ZHAO DONGXU, YIN FEI, XIAO XING, ZHANG BO-YIN, ZHU QIN GSAN; Low back pain tied to spinal endometriosis, 2014


Hedyeh Riazi, Najmeh Tehranian,corresponding author Saeideh Ziaei,corresponding author Easa Mohammadi, Ebrahim Hajizadeh, and Ali Montazeri; Clinical diagnosis of pelvic endometriosis: a scoping review; 2015; 15-39

Audebert A, Bäckström T, Barlow DH, Benagiano G, Brosens I, Bühler K, Donnez J, Evers JL, Pellicer A, Mettler L, et al.Endometriosis 1991: a discussion document.Hum Reprod. 1992 Mar;7(3):432-5.

Kitawaki J, Kado N, Ishihara H, Koshiba H, Kitaoka Y, Honjo H.Endometriosis: the pathophysiology as an estrogen-dependent disease. J Steroid Biochem Mol Biol. 2002 Dec;83(1-5):149-55.

Noble LS, Simpson ER, Johns A, Bulun SE.Aromatase expression in endometriosis.J Clin Endocrinol Metab. 1996 Jan;81(1):174-9

Brosens I, Diagnosis of endometriosis.Semin Reprod Endocrinol. 1997;15(3):229-33

rosens I, Puttemans P, Campo R, Gordts S. Diagnosis of endometriosis: Pelvic endoscopy and imaging techniques. Best Practice and Research Clinical Obstetrics and Gynaecology. 2004;18(2):285–303.

American Society for Reproductive Medicine;Revised American Society for Reproductive Medicine classification of endometriosis: 1996

Beatriz Taliberti da Costa Porto , Helizabet Salomão Abdalla Ayrosa Ribeiro , Maria Antonieta Longo Galvão , Vanessa Gozzo Sekula , José Mendes Aldrigui , Paulo Augusto Ayrosa Ribeiro.
Histological classification and quality of life in women with endometriosis .2015

Martin DC, Hubert GD, Vander Zwaag R, el-Zeky FA. Laparoscopic appearances of peritoneal endometriosis.Fertil Steril 1989; 51:63-7

Francesco Antonio Viscomi, Rogério Dias, Laurival de Luca, Mauro Fernando Kürten Ihlenfeld. Correlation Between Laparoscopic Aspects and Histologic Findings in Peritoneal Endometriotic Lesions. Rev. Bras. Ginecol. Obstet. v.24 2002 .

Ulukus M, Arici A.Immunology of endometriosis.Minerva Ginecol. 2005 Jun;57(3):237-48.

Sinaii N, Cleary SD, Ballweg ML, Nieman LK, Stratton P.High rates of autoimmune and endocrine disorders, fibromyalgia, chronic fatigue syndrome and atopic diseases among women with endometriosis: a survey analysis.Hum Reprod. 2002 Oct;17(10):2715-24.

Christodoulakos G, Augoulea A, Lambrinoudaki I, Sioulas V, Creatsas G. Pathogenesis of endometriosis: the role of defective 'immunosurveillance'.Eur J Contracept Reprod Health Care. 2007 Sep;12(3):194-202.

Nisolle M, Donnez J. "Peritoneal endometriosis, ovarian endometriosis, and
adenomyotic nodules of the rectovaginal septum are three different entities."
Fertil Steril. 1997;68(4):585-96.

Redwine DB1.Age-related evolution in color appearance of endometriosis.Fertil Steril. 1987 Dec;48(6):1062-3.

Koninckx PR, Martin DC,Deep endometriosis: a consequence of infiltration or retraction or possibly adenomyosis externa?Fertil Steril. 1992 Nov;58(5):924-8.

Acosta AA, Buttram VC, Jr., Besch P K, Malinak LR, Franklin RR,
Vanderheyden JD. "A proposed classification of pelvic endometriosis."
Obstet Gynecol. 1973;42(1):19-25.


Arruda MS1, Petta CA, Abrão MS, Benetti-Pinto CL.Time elapsed from onset of symptoms to diagnosis of endometriosis in a cohort study of Brazilian women. Hum Reprod. 2003 Apr;18(4):756-9.

Hornstein MD1, Yuzpe AA, Burry KA, Heinrichs LR, Buttram VL Jr, Orwoll ES. Prospective randomized double-blind trial of 3 versus 6 months of nafarelin therapy for endometriosis associated pelvic pain.Fertil Steril. 1995 May;63(5):955-62.

Stephen Kennedy , Agneta Bergqvist  , Charles Chapron  , Thomas D’Hooghe  , Gerard Dunselman  , Robert Greb  , Lone Hummelshoj  , Andrew Prentice  , Ertan Saridogan .ESHRE guideline for the diagnosis and treatment of endometriosis. 2005.

Bazot M1, Detchev R, Cortez A, Amouyal P, Uzan S, Daraï E.Transvaginal sonography and rectal endoscopic sonography for the assessment of pelvic endometriosis: a preliminary comparison.Hum Reprod. 2003 Aug;18(8):1686-92. (A)

Marc Bazot, Emile Darai, Roula Hourani, Isabelle Thomassin, Annie Cortez, Serge Uzan, 
Jean-Noe l Buy, MDDeep Pelvic Endometriosis: MR Imaging for Diagnosis and Prediction of Extension of Disease 1 (B).


Moore J, Copley S, Morris J, Lindsell D, Golding S, Kennedy S. A .systematic review of the accuracy of ultrasound in the diagnosis of endometriosis. Ultrasound Obstet Gynecol. 2002 .

Fedele L1, Bianchi S, Raffaelli R, Portuese A. Pre-operative assessment of bladder endometriosis. Hum Reprod 1997; 12:2519 –2522.

Kinkel K1, Chapron C, Balleyguier C, Fritel X, Dubuisson JB, Moreau JF.. Magnetic resonance imaging characteristics of deep endometriosis. Hum Reprod 1999; 14:1080 –1086.

Fleischer AC, Cullinan JW, Walsch JW. Problem-oriented gynecologic imaging with emphasis on usltrasonography. In Fleischer AC, Manning FA, Jeanty P, Romero R.(Ed.). Sonography in Obstetrics and Gynecology. A.
Lange. 1996;887.

Abrão MS, Gonçalves MO, Dias JA Jr, Podgaec S, Chamie LP, Blasbalg R. Comparison between clinical examination, transvaginal sonography and magnetic resonance imaging for the diagnosis of
deep endometriosis. Hum Reprod. 2007;22(12):3092-7


Jacobson TZ, Barlow DH, Koninckx PR, Olive D, Farquhar C. Laparoscopic surgery for subfertility associated with endometriosis. Cochrane Database Syst Rev. 2002;(4):CD001398.


Chapron C, Vercellini P, Barakat H, Vieira M, Dubuisson JB. Management of ovarian endometriomas. Hum Reprod Update. 2002;8(6):591-7.

Adamson GD, Nelson HP. Surgical treatment of endometriosis. Obstet Gynecol Clin North Am 1997; 24:375– 409.



Júlia Kefalás Troncon, Ana Carolina Tagliatti Zani, Andrea Duarte Damasceno Vieira, Omero Benedicto Poli-Neto, Antônio Alberto Nogueira, and Júlio César Rosa-e-Silva .Endometriosis in a Patient with Mayer-Rokitansky-Küster-Hauser Syndrome . Case Rep Obstet Gynecol. 2014

Matthew Latham Macer, M.D. and Hugh S. Taylor. Endometriosis and Infertility: A review of the pathogenesis and treatment of endometriosis-associated infertility 2013.

A.G. Braundmeier and A.T. Fazleabas. The non-human primate model of endometriosis: research and implications for fecundity. 2009.

Jessica Fourquet, Xin Gao, Diego Zavala, Juan C. Orengo, Sonia Abac, Abigail Ruiz, Joaquín Laboy, and Idhaliz Flores Patients’ report on how endometriosis affects health, work, and daily life, 2011.


ADOLFO PISANU, DANIELA DEPLANO, STEFANO ANGIONI, ROSSANO AMBU E ALESSANDRO UCCHEDDU. Rectal perforation from endometriosis in pregnancy: Case report and literature review, World J. Gastroenterol. 2010 .

Gao X, Yeh YC, Outley J, Simon J, Botteman M, Spalding J. Health-related quality of life burden of women with endometriosis: a literature review. Curr Med Res Opin. 2006.

S. Kundu, J. Wildgrube, C. Schippert, P. Hillemanns, and I. Brandes.Supporting and Inhibiting Factors When Coping with Endometriosis From the Patients Perspective. Geburtshilfe Frauenheilkd. 2015.

Süha Akpınar, Güliz Yılmaz,corresponding author and Emre Çelebioglu. A rare cyclic recurrent hematuria case; bladder endometriosis.Quant Imaging Med Surg. 2015 

Huirne JA, Lambalk CB.Gonadotropin-releasing-hormone-receptor antagonists.Lancet. 2001.

Wellbery C. Diagnosis and treatment of endometriosis. American Family Physician. 1999;60(6):1753–1762.

Leticia Muñoz-Hernando, Jose L Muñoz-Gonzalez, Laura Marqueta-Marques, Carmen Alvarez-Conejo, Álvaro Tejerizo-García, Gregorio Lopez-Gonzalez, Emilia Villegas-Muñoz, Angel Martin-Jimenez, and Jesús S Jiménez-López. Endometriosis: alternative methods of medical treatment.Int J Womens Health. 2015; 7: 595–603. 

Schrager S, Falleroni J, Edgoose J. Evaluation and treatment of endometriosis.Am Fam Physician. 2013

Gabriella Zito, Stefania Luppi, Elena Giolo, Monica Martinelli, Irene Venturin, Giovanni Di Lorenzo, and Giuseppe Ricci .Medical Treatments for Endometriosis-Associated Pelvic Pain.Biomed Res Int. 2014.

Cottreau CM, Ness RB, Modugno F, Allen GO, Goodman MT. Endometriosis and its treatment with danazol or lupron in relation to ovarian cancer. Clin Cancer Res. 2003.

Fariba Almassinokiani, Alireza Almasi, Peyman Akbari, and Mahboubeh Saberifard.Effect of Letrozole on endometriosis-related pelvic pain.Med J Islam Repub Iran. 2014; 28: 107. 

Betul Usluogullari,corresponding author Candan Zehra Duvan, and Celil Alper Usluogullari.Use of aromatase inhibitors in practice of gynecology.J Ovarian Res. 2015.

Guo SW.Recurrence of endometriosis and its control.Hum Reprod Update. 2009 Jul-Aug;15(4):441-61.




\end{document}
